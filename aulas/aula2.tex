\title{Linguagens de Marcação e Introdução ao HTML5}
\date{\today}
\frame{\titlepage}

% Slide: O que é HTML?
% Slide 1: A Origem da Internet e o Nascimento do HTML
\begin{frame}[fragile]
  \frametitle{A Origem da Internet e o Nascimento do HTML}
  \begin{itemize}
    \item A internet começou como um projeto de pesquisa nos anos 1960, tornando-se uma ferramenta pública na década de 1990.
    \item O HTML foi criado por Tim Berners-Lee no início dos anos 1990 como uma forma de estruturar e vincular documentos.
    \item Sua criação marcou o início da World Wide Web, permitindo a criação de páginas acessíveis globalmente através de navegadores.
  \end{itemize}
\end{frame}

% Slide 2: Por que HTML?
\begin{frame}[fragile]
  \frametitle{Por que HTML?}
  \begin{itemize}
    \item HTML simplifica a apresentação de informações na internet, introduzindo uma maneira padronizada de estruturar texto e multimídia.
    \item Ele permite a interligação entre documentos por meio de hiperlinks, formando a teia de conexões que define a web.
    \item A facilidade de uso e o amplo suporte em navegadores tornaram o HTML a espinha dorsal da criação de conteúdo na web.
  \end{itemize}
\end{frame}

% Slide 3: A Evolução do HTML e sua Importância Atual
\begin{frame}[fragile]
  \frametitle{A Evolução do HTML e sua Importância Atual}
  \begin{itemize}
    \item Desde sua criação, o HTML evoluiu através de várias versões, incorporando novos elementos e funcionalidades para atender às crescentes demandas da web moderna.
    \item HTML5, a versão mais recente, inclui suporte para gráficos, áudio, vídeo, e interações complexas, abrindo novas possibilidades para desenvolvedores e criadores de conteúdo.
    \item Entender HTML é essencial para qualquer pessoa que deseja criar ou gerenciar conteúdo na internet, sendo uma habilidade fundamental no campo do desenvolvimento web.
  \end{itemize}
\end{frame}
\begin{frame}[fragile]
  \frametitle{HTML}
  \begin{figure}
    \centering
    \includegraphics[width=0.8\textwidth]{assets/tatoo.jpeg}
  \end{figure}
\end{frame}

% Slide: Conteúdos do <head>
\begin{frame}[fragile]
  \frametitle{Conteúdos da Seção \texttt{<head>}}
  \begin{itemize}
    \item \textbf{Título (\texttt{<title>}):} Define o título da página, que aparece na aba do navegador.
    \item \textbf{Meta Tags (\texttt{<meta>}):} Fornecem metadados sobre a página HTML, como conjunto de caracteres, descrição da página, e palavras-chave.
      \begin{itemize}
        \item Exemplo: \texttt{<meta charset="UTF-8">} especifica a codificação de caracteres para a página.
      \end{itemize}
    \item \textbf{Links para Folhas de Estilo (\texttt{<link>}):} Referencia folhas de estilo externas (CSS) que definem a aparência da página.
      \begin{itemize}
        \item Exemplo: \texttt{<link rel="stylesheet" href="styles.css">}
      \end{itemize}
    \item \textbf{Scripts (\texttt{<script>}):} Inclui código JavaScript que pode ser executado para adicionar interatividade à página.
      \begin{itemize}
        \item Exemplo: \texttt{<script src="script.js"></script>}
      \end{itemize}
  \end{itemize}
\end{frame}

% Slide: Conteúdos do <body>
\begin{frame}[fragile]
  \frametitle{Conteúdos da Seção \texttt{<body>}}
  \begin{itemize}
    \item A seção \texttt{<body>} contém o conteúdo principal da página, visível aos usuários.
    \item \textbf{Textos e Parágrafos (\texttt{<h1>}-\texttt{<h6>}, \texttt{<p>}):} Usados para definir cabeçalhos e parágrafos.
    \item \textbf{Imagens (\texttt{<img>}):} Incorpora imagens na página.
      \begin{itemize}
        \item Exemplo: \texttt{<img src="imagem.jpg" alt="Descrição">}
      \end{itemize}
    \item \textbf{Links (\texttt{<a>}):} Cria hiperlinks para outras páginas.
      \begin{itemize}
        \item Exemplo: \texttt{<a href="outra\_pagina.html">Visite outra página</a>}
      \end{itemize}
    \item \textbf{Listas (\texttt{<ul>}, \texttt{<ol>}, \texttt{<li>}):} Organiza informações em listas não ordenadas ou ordenadas.
    \item \textbf{Formulários (\texttt{<form>}):} Permite a coleta de dados do usuário.
      \begin{itemize}
        \item Exemplo: \texttt{<form action="/submit" method="post">...</form>}
      \end{itemize}
    \item \textbf{Vídeos e Áudios (\texttt{<video>}, \texttt{<audio>}):} Incorpora conteúdo de mídia.
  \end{itemize}
\end{frame}


% Slide: Estrutura Básica do HTML
\begin{frame}[fragile]
  \frametitle{Estrutura Básica do HTML}
  \begin{block}{Exemplo de Estrutura Básica}
    \begin{verbatim}
<!DOCTYPE html>
<html>
<head>
    <title>Título da Página</title>
</head>
<body>
    <h1>Meu Primeiro Título</h1>
    <p>Meu primeiro parágrafo.</p>
</body>
</html>
    \end{verbatim}
  \end{block}
  \begin{itemize}
    \item Toda página HTML começa com a declaração do tipo de documento: \texttt{<!DOCTYPE html>}.
    \item A estrutura é dividida em \texttt{<head>} (cabeçalho) e \texttt{<body>} (corpo).
  \end{itemize}
\end{frame}

% Slide: Formulários em HTML
\begin{frame}[fragile]
  \frametitle{Formulários em HTML}
  \begin{itemize}
    \item Formulários são utilizados para coletar dados dos usuários.
    \item Compostos por \texttt{<form>}, \texttt{<input>}, \texttt{<label>}, entre outros.
    \item Exemplo prático de um formulário simples:
    \begin{block}{Exemplo de Formulário}
    \begin{verbatim}
<form action="/submit" method="post">
  <label for="name">Nome:</label>
  <input type="text" id="name" name="name">
  <input type="submit" value="Enviar">
</form>
    \end{verbatim}
    \end{block}
  \end{itemize}
\end{frame}

% Slide: Links e Multimídia
\begin{frame}[fragile]
  \frametitle{Links e Multimídia}
  \begin{itemize}
    \item \textbf{Links:} Elemento \texttt{<a>} para criar hiperlinks.
    \item \textbf{Imagens:} Elemento \texttt{<img>} para incorporar imagens.
    \item \textbf{Vídeos e Áudios:} Elementos \texttt{<video>} e \texttt{<audio>} para adicionar mídias.
  \end{itemize}
\end{frame}

% Slide: Atividade Prática
\begin{frame}[fragile]
  \frametitle{Atividade Prática}
  \textbf{Objetivo:} Criar uma página simples utilizando HTML5.
  \begin{itemize}
    \item Inclua título, parágrafos, uma imagem e um formulário.
    \item Experimente com links para outras páginas.
    \item Tente criar um formulário simples.
  \end{itemize}
  \textbf{Entrega:} Submeta o arquivo HTML através da plataforma de ensino.
\end{frame}

\begin{frame}[fragile]
  \frametitle{Iniciando com FrontEditor}
  \begin{itemize}
    \item Para os primeiros passos no desenvolvimento web com HTML e CSS, recomendamos o uso do \textbf{FrontEditor}.
    \item \textbf{Por que FrontEditor?}
      \begin{itemize}
        \item Interface simples e intuitiva, ideal para iniciantes.
        \item Permite visualização imediata das alterações no código.
        \item Não requer instalação, funcionando diretamente no seu navegador.
        \item Ótimo para experimentar conceitos básicos e ver o resultado em tempo real.
      \end{itemize}
    \item \textbf{Acesso:} Você pode começar a usar agora mesmo acessando \url{https://www.fronteditor.dev/}
    \item À medida que você se familiariza com HTML e CSS, a transição para um ambiente de desenvolvimento mais avançado como o \textbf{Visual Studio Code} será natural e proveitosa.
  \end{itemize}
\end{frame}

% Slide: Estrutura Básica do HTML
\begin{frame}[fragile]
  \frametitle{Estrutura Básica do HTML}
  \begin{block}{Exemplo de Estrutura Básica}
    \small
    \begin{verbatim}
      <!DOCTYPE html>
      <html>
      <head>
          <title>Minha Primeira Página</title>
      </head>
      <body>
          <h1>Bem-vindo à Minha Página Web</h1>
          <p>Este é um parágrafo de introdução sobre esta página.</p>
          <h2>Sobre Mim</h2>
          <p>Meu nome é [Seu Nome] e estou aprendendo HTML.</p>
      </body>
      </html>
      
    \end{verbatim}
  \end{block}
\end{frame}

% Slide: Estrutura Básica do HTML
\begin{frame}[fragile]
  \frametitle{Estrutura Básica do HTML}
  \begin{block}{Exemplo de Estrutura Básica}
    \small
    \begin{verbatim}
      <!DOCTYPE html>
      <html>
      <head>
          <title>Links e Imagens</title>
      </head>
      <body>
          <h1>Links Úteis</h1>
          <p>Visite o <a href="https://www.wikipedia.org/">Wikipedia</a> para mais informações.</p>
          <h2>Imagem Exemplar</h2>
          <p>Aqui está uma imagem interessante:</p>
          <img src="https://example.com/imagem.jpg" alt="Descrição da Imagem">
      </body>
      </html>

    \end{verbatim}
  \end{block}
\end{frame}

% Slide: Estrutura Básica do HTML
\begin{frame}[fragile]
  \frametitle{Estrutura Básica do HTML}
  \begin{block}{Exemplo de Estrutura Básica}
    \small
    \begin{verbatim}
      <!DOCTYPE html>
      <html>
      <head>
          <title>Formulário Simples</title>
      </head>
      <body>
          <h1>Formulário de Contato</h1>
          <form method="post">
              <label for="name">Nome:</label><br>
              <input type="text" id="name" name="name"><br>
              <label for="email">Email:</label><br>
              <input type="email" id="email" name="email"><br>
              <input type="submit" value="Enviar">
          </form>
      </body>
      </html>
      
    \end{verbatim}
  \end{block}
\end{frame}
\begin{frame}[fragile]
  \frametitle{Referências de Documentação do HTML}
  \begin{itemize}
    \item \textbf{MDN Web Docs (Mozilla Developer Network):}
      \begin{itemize}
        \item Uma das fontes mais abrangentes e confiáveis para aprender sobre HTML, CSS, e JavaScript.
        \item Endereço: \url{https://developer.mozilla.org/en-US/docs/Web/HTML}
      \end{itemize}
    \item \textbf{W3Schools:}
      \begin{itemize}
        \item Oferece tutoriais fáceis de seguir, referências e exemplos para web development.
        \item Endereço: \url{https://www.w3schools.com/html/}
      \end{itemize}
    \item \textbf{HTML Living Standard:}
      \begin{itemize}
        \item Mantido pelo WHATWG (Web Hypertext Application Technology Working Group), documenta o padrão atual do HTML.
        \item Endereço: \url{https://html.spec.whatwg.org/}
      \end{itemize}
    \item \textbf{HTML5 Rocks:}
      \begin{itemize}
        \item Um recurso projetado para desenvolvedores avançados, oferecendo artigos e tutoriais sobre as mais recentes tecnologias da web.
        \item Endereço: \url{http://www.html5rocks.com/}
      \end{itemize}
  \end{itemize}
\end{frame}

\begin{frame}[fragile]
  \frametitle{Introdução ao CSS}
  \begin{itemize}
    \item \textbf{O que é CSS?} CSS (Cascading Style Sheets) é a linguagem usada para estilizar e layout de páginas web. Trabalha em conjunto com o HTML, que estrutura o conteúdo.
    \item \textbf{Origens do CSS:} Foi proposto pela primeira vez por Håkon Wium Lie em 1994. Desde então, evoluiu para se tornar o padrão de estilização na web.
    \item \textbf{Propósito:} Permite aos desenvolvedores e designers controlar a aparência do site, incluindo cores, fontes, espaçamentos, layout e muito mais, em diferentes dispositivos e tamanhos de tela.
  \end{itemize}
\end{frame}

\begin{frame}[fragile]
  \frametitle{Estrutura Básica do CSS}
  \begin{itemize}
    \item O CSS pode ser aplicado diretamente nas tags HTML, dentro de um elemento \texttt{<style>} no \texttt{<head>}, ou em um arquivo externo.
    \item Uma regra CSS consiste em um seletor e um bloco de declaração: \texttt{seletor { propriedade: valor; }}
    \item \textbf{Seletor:} Identifica os elementos HTML a serem estilizados.
    \item \textbf{Declaração:} Define como os elementos selecionados devem ser estilizados.
  \end{itemize}
\end{frame}

\begin{frame}[fragile]
  \frametitle{ID vs. Classe}
  \begin{itemize}
    \item \textbf{ID (\texttt{\#}):} Identificador único para um elemento HTML. Cada elemento pode ter apenas um ID, e cada página deve ter IDs únicos.
      \begin{itemize}
        \item Exemplo: \texttt{\#navbar { ... }}
      \end{itemize}
    \item \textbf{Classe (\texttt{.}):} Pode ser usada em vários elementos dentro de uma página. Um elemento pode ter várias classes.
      \begin{itemize}
        \item Exemplo: \texttt{.highlight { ... }}
      \end{itemize}
    \item A escolha entre ID e classe depende do objetivo: ID para estilização única, Classe para estilização compartilhada ou repetida.
  \end{itemize}
\end{frame}

\begin{frame}[fragile]
  \frametitle{Referências de Documentação do CSS}
  \begin{itemize}
    \item \textbf{MDN Web Docs (Mozilla Developer Network):}
      \begin{itemize}
        \item Fornece uma documentação abrangente sobre CSS, cobrindo desde fundamentos até técnicas avançadas.
        \item Endereço: \url{https://developer.mozilla.org/en-US/docs/Web/CSS}
      \end{itemize}
    \item \textbf{CSS-Tricks:}
      \begin{itemize}
        \item Um site dedicado a dicas, truques e técnicas de CSS. Ótimo para soluções específicas e inspiração de design.
        \item Endereço: \url{https://css-tricks.com/}
      \end{itemize}
    \item \textbf{W3Schools CSS Tutorial:}
      \begin{itemize}
        \item Oferece uma introdução passo a passo ao CSS, ideal para iniciantes e também como referência rápida para desenvolvedores experientes.
        \item Endereço: \url{https://www.w3schools.com/css/}
      \end{itemize}
    \item \textbf{A List Apart:}
      \begin{itemize}
        \item Explora o design, o desenvolvimento e o significado do conteúdo web, com um foco particular em padrões web e melhores práticas.
        \item Endereço: \url{https://alistapart.com/}
      \end{itemize}
  \end{itemize}
\end{frame}
