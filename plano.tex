
PROGRAMA
Ciências Exatas e da Terra / Ciência da Computação / Metodologia e Técnicas da Computação / Engenharia de Software
1. Engenharia de Requisitos: conceitos, métodos e ferramentas.
2. Análise e Projeto (Design) de Software: princípios, modelos e linguagens de modelagem de software e ferramentas.
3. Métodos de Desenvolvimento Tradicionais e Ágeis: conceitos, diferenças, características, Processo Unificado/RUP, Processo Unificado Ágil, XP, Scrum.
4. Gerência de Configuração de Software: gerenciamento de mudanças e controle de versão, integração contínua, entrega contínua e implantação contínua.
5. Arquitetura de Software: definições, estilos arquiteturais, notações arquiteturais.
6. Manutenção e evolução de software: conceitos e terminologia, processos e atividades, compreensão de software.
7. Medição de Software: conceitos, métodos e procedimentos.
8. Qualidade do Produto: conceitos, padrões e normas, métricas.
9. Verificação, Validação e Teste de Software: conceitos, técnicas, ferramentas.
10. Técnicas de modelagem de processos de negócio: BPMN e UML.
BIBLIOGRAFIA BÁSICA
BOOCH, G; RUMBAUGH, J. JACOBSON, I. UML: Guia do Usuário. Editora Elsevier, 2a edição, 2006.
DELAMARO, ME; MALDONADO, JC; JINO, M. Introdução ao Teste de Software. Editora Elsevier, 1a edição, 2007.
GAMMA, E; HELM, R; JOHNSON, R; VLISSIDES, J. Padrões de Projeto – Soluções reutilizáveis de software orientado a objetos. Editora Bookman, 2000. LARMAN, G. Utilizando UML e Padrões. Editora Bookman, 3a edição, 2008.
MENS, Tom; SEREBRENIK, Alexander; CLEVE, Anthony (Ed.). Evolving Software Systems. Heidelberg: Springer, C2014. Xxiii, 404 P. Isbn 9783642453977
SOMMERVILLE, I. Engenharia de Software. Editora McGrawHill, 9a edição, 2011.
PRESSMAN, Roger S. Engenharia de Software: Uma Abordagem Profissional. 8. Ed. Porto Alegre, Rs: Amgh Ed., 2016. Xxviii, 940 P. Isbn 9788580555332.
GHEZZI, C. Fundamentals of Software Engineering. Prentice Hall, 2003.
ROCHA, A.; SANTOS, Gleison; BARCELLOS, Monalessa. (2012). Medição de Software e Controle Estatístico de Processos. Ministério de Ciência e Tecnologia. Disponibilizado pelos autores em https://nemo.inf.ufes.br/wp-content/uploads/Monalessa/LivroMedicao&CEP_RochaSantosBarcellos_2012.pdf.
SHORE, J; WARDEN, S. The art of Agile Development. O ́Reilly, 2008.
WAZLAWICK, R. S. Engenharia de Software, conceitos e práticas, Editora Campus, 2013.
VALLE, R.; OLIVEIRA, S. B. Análise e Modelagem de Processos de Negócio: Foco na Notação BPMN. São Paulo: Atlas, 2009.


Plano de estudos:
Julho (22-31) - Engenharia de Requisitos
Estudo: Capítulos relevantes nos livros de SOMMERVILLE, PRESSMAN e WAZLAWICK.
Plano de aula: Deve abordar conceitos, métodos e ferramentas de engenharia de requisitos.
Agosto (1-10) - Análise e Projeto (Design) de Software
Estudo: BOOCH, RUMBAUGH, JACOBSON, GAMMA, HELM, JOHNSON, VLISSIDES e LARMAN.
Plano de aula: Princípios, modelos e linguagens de modelagem de software e ferramentas.
Agosto (11-20) - Métodos de Desenvolvimento Tradicionais e Ágeis
Estudo: Capítulos relevantes nos livros de PRESSMAN, SHORE & WARDEN e WAZLAWICK.
Plano de aula: Conceitos, diferenças, características, Processo Unificado/RUP, Processo Unificado Ágil, XP, Scrum.
Agosto (21-31) - Gerência de Configuração de Software
Estudo: Capítulos relevantes nos livros de SOMMERVILLE, PRESSMAN e WAZLAWICK.
Plano de aula: Gerenciamento de mudanças e controle de versão, integração contínua, entrega contínua e implantação contínua.
Setembro (1-10) - Arquitetura de Software
Estudo: GHEZZI, GAMMA, HELM, JOHNSON, VLISSIDES e LARMAN.
Plano de aula: Definições, estilos arquiteturais, notações arquiteturais.
Setembro (11-20) - Manutenção e evolução de software
Estudo: MENS, SEREBRENIK, CLEVE.
Plano de aula: Conceitos e terminologia, processos e atividades, compreensão de software.
Setembro (21-30) - Medição de Software
Estudo: ROCHA, SANTOS, BARCELLOS.
Plano de aula: Conceitos, métodos e procedimentos de medição de software.
Outubro (1-10) - Qualidade do Produto e Verificação, Validação e Teste de Software
Estudo: DELAMARO, MALDONADO, JINO, SOMMERVILLE, PRESSMAN.
Plano de aula: Conceitos, padrões e normas, métricas, técnicas, ferramentas de teste.
Outubro (11-16) - Técnicas de modelagem de processos de negócio
Estudo: VALLE, OLIVEIRA.
Plano de aula: BPMN e UML.
Outubro (17-21) - Revisão Geral

Revise todos os tópicos e concentre-se em áreas que você considera mais desafiadoras.
Lembre-se de incluir tempo para exercícios práticos e resolução de questões de provas anteriores. Isso te ajudará a consolidar o conteúdo e a estar mais preparado para o tipo de perguntas que podem ser feitas. Boa sorte nos estudos!
